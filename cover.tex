\quad \vspace{1cm}
\begin{center}
\includegraphics[width=2in]{image/fjcu.png} \\
\vspace{1cm}
\begin{bfseries}
\Large{FJCU 自辦賽} \\
\end{bfseries}
% \small{2020 Spring}
\end{center}
\vspace{1cm}
\begin{center}
    \begin{tabular}{clc}
        \toprule
            ID & Problem Name & Time Limit \\
        \midrule
            A & 題目名稱 & 1 sec \\
        \bottomrule
    \end{tabular}
\end{center}
\thispagestyle{empty}
\clearpage  % 清除當頁頁碼
\quad
\subsection*{規則}
任何違反規則的行為將被視為作弊,並取消考試資格。
\begin{enumerate}
    \item 除了紙筆和身份證件和水,其餘物品請放包包,或交給監考人員。計算紙由監考方提供。
    \item 不得離開座位。若有需要上廁所或拿必須用品,必須舉手取得監考人員同意,方可離座。
    \item 不得與他人研討試題。
    \item 題目共 7 題,全英文。
    \item 只能使用提供的 PC2 帳號進行考試。
    \item 如果您有任何關於題目或是和考試相關的問題,請舉手詢問監考人員。 不要問或與其他人討論。
    \item 禁止任何干擾考試的行為。
\end{enumerate}
\newpage
\subsection*{Rules}
Any violation of the rules is considered cheating and will result in a disqualification.
\begin{enumerate}
    \item Except for paper and pen, ID card and water, please put the other items in your bag or hand it to the invigilator. The calculation paper is provided by the invigilator.
    \item You can't leave their seats. If you need to go to the toilet or take necessary supplies, you must raise your hand to obtain the consent of the invigilator before leaving the seat.
    \item You must not discuss problems or share ideas/solutions with any other people.
    \item There are 7 questions in total, all in English.
    \item You should take place in the exam using your personal and only PC2 account given prior to the exam.
    \item Should you have a question about the problems or face any exam-related issue, please raise your hand and ask the invigilator. Do not ask or discuss with any other people.
    \item Any malicious action interfering the exam is prohibited.
\end{enumerate}
\newpage
\subsection*{分數}
\begin{enumerate}
    \item 必須經由 PC2 上傳程式,評測系統只回應在測驗期間 (180 分鐘) 上傳得程式碼。 對於每筆提交,回應內容會是下列其中之一。
        \begin{itemize}
            \item \texttt{Yes}: 程式碼正確。
            \item \texttt{Compiler-Error}: 程式碼無法成功編譯。
            \item \texttt{TimeLimit}: 程式花太多的時間。
            \item \texttt{Run-Error}: 程式回傳非 0 的結果,通常代表你的程式被作業系統停止。
            \item \texttt{Wrong-Answer}: 程式輸出結果不正確。
            \item \texttt{No-Output}: 程式沒有任何輸出。
        \end{itemize}
    \item 成績會依據正確解出之題數,如下表。
        \begin{center}
            \begin{tabular}{cc}
                \toprule
                    解出題數 & 成績 \\
                \midrule
                    0 & 0 \\
                    1 & 35 \\
                    2 & 65 \\
                    3 & 80 \\
                    4 & 90 \\
                    5 & 95 \\
                    6 & 98 \\
                    7 & 100 \\
                \bottomrule
            \end{tabular}
        \end{center}
\end{enumerate}
\newpage
\subsection*{Scoring}
\begin{enumerate}
    \item You must submit your solutions via DOMjudge. The judge system will only respond to submissions that are submitted within the exam duration (180 minutes). The response to each run must be one of the following:
        \begin{itemize}
            \item \texttt{Yes}: The judge accepts your code.
            \item \texttt{Compiler-Error}: Your code cannot be successfully compiled.
            \item \texttt{TimeLimit}: Your program consumes too much time.
            \item \texttt{Run-Error}: Your program terminates with an non-zero return code, which often means your program is terminated by the operating system.
            \item \texttt{Wrong-Answer}: The judge rejects the output of your program.
            \item \texttt{No-Output}: Your program does not generate any output.
        \end{itemize}
    \item The score you get for final exams is based on the total number of problems to which a correct solution is submitted:
        \begin{center}
            \begin{tabular}{cc}
                \toprule
                    Solved & Score \\
                \midrule
                    0 & 0 \\
                    1 & 35 \\
                    2 & 65 \\
                    3 & 80 \\
                    4 & 90 \\
                    5 & 95 \\
                    6 & 98 \\
                    7 & 100 \\
                \bottomrule
            \end{tabular}
        \end{center}
\end{enumerate}
% \subsection*{Hint}
% The problems are sorted by their names (lexicographically) and not by increasing difficulty; problems that appear first are not necessarily easier. Because of this, it is recommended that you read every problem.
\clearpage  % 清除當頁頁碼
